\chapter{Git-Tipps}
Ein Schreckensszenario für alle Studenten ist es, Teile oder gar die ganze Abschlussarbeit zu verlieren, weil das Speichermedium kaputt geht oder die Datein ausversehen gelöscht oder überschrieben werden.
Nun weiß hoffentlich jeder theoretisch, dass Backups eine gute Idee sind aber die Frage ist, ob man das dann praktisch auch jede halbe Stunde macht und ob man dann hunderte Dateien wie \enquote{Bachelorarbeit-überarbeitet-v125-final-jetztwirklichfinal-needochnicht-aberjetzt.docx} im Email-Posteingang hat und dann doch die falsche Version wiederherstellt.
Daher gehe ich noch einen Schritt weiter und empfehle euch und Ihnen, die Abschlussarbeit in ein Git-Repository z.\,B. auf GitHub oder das Informatik GitLab zu stellen.
Dies hat mehrere Vorteile:

\begin{enumerate}
\item hat man natürlich ein Backup, und das auf einem anderen System, und ohne seinen Rechner mit Kopien vollzumüllen.
\item kann man nicht nur den letzten Stand sondern \emph{jeden} Commit wiederherstellen und sogar verschiedene Stände zusammenfügen.
\item sieht man dank der textbasierten LaTeX-Formate wie .tex und .bib in den Git-diffs genau, welche Änderungen wann durchgeführt wurden.
\item kann man mit Features wie Issues seine Arbeit organisieren und durch automatische Emails bei passenden Commits die Kommunikation mit den Betreuern vereinfachen.
\item erleichtert es das Arbeiten an verschiedenen Geräten, auch wenn es mal kein Internet gibt.
\end{enumerate}

\section{GitHub und GitLab}
Auf GitHub kann man sich einfach einen Account auf \url{https://github.com/} anlegen.
Alternativ kann man als Informatikstudent auch einen Account auf \url{https://git.informatik.uni-leipzig.de/} anlegen, die grundlegenden Features sind ähnlich.
Die browserbasierte LaTeX-Umgebung \href{https://www.overleaf.com/}{OverLeaf} bietet auch eine Git-Integration.

Git-Tutorials gibt es z.\,B. unter \url{https://git-scm.com/docs/gittutorial} und \url{https://www.vogella.com/tutorials/Git/article.html}.
Besonders tief muss man aber gar nicht in die Materie einsteigen, im Alltag benutzt man fast nur git clone, status, pull, add, commit und push.
