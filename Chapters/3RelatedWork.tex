%*****************************************
\chapter{Stand der Forschung}\label{ch:relatedWork}
%*****************************************

Kann auch im Grundlagenkapitel mit integriert werden! Wenn es zwei Kapitel sind, ist das Grundlagenkapitel für bekanntes \enquote{Lehrbuchwissen} gedacht, der \enquote{Stand der Forschung} für Vorarbeiten in Publikationen.

Grundlage für eine Abschlussarbeit ist eine gründliche Recherche im Themenumfeld.
Dabei ist es ausdrücklich nicht hinreichend, mit bekannten Suchmaschinen im Internet zu recherchieren.
Vielmehr wird von den Studierenden erwartet, dass sie auch referierte Veröffentlichungen (wissenschaftliche Zeitschriften (auch elektronisch), Bücher) in die Erarbeitung einbeziehen (und mit entsprechenden Quellenangaben belegen).
Informationen zum Thema der Literaturrecherche finden sich in den \href{http://www.imise.uni-leipzig.de/Lehre/MedInf/Abschlussarbeiten/Literaturrecherche}{Hinweisen zur Literaturrecherche}.
Wir empfehlen Ihnen, Kurse zur Literaturrecherche zu belegen, z.B. beim \href{https://home.uni-leipzig.de/academiclab/}{Academic Lab} oder der \href{https://www.ub.uni-leipzig.de/service/workshops-und-online-tutorials/}{Universitätsbibliothek}.
Im \href{https://home.uni-leipzig.de/schreibportal/}{Onlineportal zum Wissenschaftlichen Schreiben der Uni Leipzig} finden Sie wertvolle Hinweise und Übungen zu Textstruktur, Stilistik, Schreibprozess sowie Quellen und Zitaten.

\paragraph{Beispielzitierungen}
\citet{sniktec} beschreiben ein Verfahren zur X von Y auf Basis von Z.
Alternativ: X von Y lässt sich auf Basis von Z ermitteln~\citep{sniktec}.

