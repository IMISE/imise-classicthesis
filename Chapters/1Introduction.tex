%************************************************
\chapter{Einleitung}\label{ch:introduction}
%************************************************
Eine Abschlussarbeit ist mit einem Projekt vergleichbar und in der Einleitung wird die Aufgabenstellung in ähnlicher Weise beschrieben wie in dem Strukturplan eines Projektes.
Daher sollte die Einleitung mit Hilfe der \enquote{5-Stufen-Methode zur systematischen Strukturplanung}\footnote{}1 vorgenommen werden.

\section{Gegenstand und Motivation}
Dieses Kapitel hilft nicht nur dem Leser, sondern auch dem Autor zu verstehen, in welchen fachlichen Kontext sich die Arbeit einordnet. Hier muss also insgesamt deutlich werden, wozu die in der Arbeit beschriebenen Arbeitsergebnisse letztlich verwendet werden sollen und welchen Nutzen sie möglicherweise bringen können.
Ohne ausreichende Gegenstands- und Motivationsbeschreibung kann ein Leser nicht verstehen, warum z.B. eine entwickelte Software sinnvoll ist bzw. zur Lösung welches Problems sie verwendet werden soll. Gerade für die Medizinische Informatik als eine problemorientierte Disziplin ergibt sich der Wert einer Lösung, z.B. einer Software, aber vor allem daraus, ob bzw. wie weit sie ein Problem löst.
Für den Autor bedeutet daher eine unzureichende Gegenstands- und Motivationsbeschreibung die Gefahr, dass er sich die zu lösende Problematik nicht ausreichend klar gemacht hat.
Bei der Erstellung der Arbeit besteht dann die Gefahr, dass er möglicherweise methodisch aufregende Lösungen entwirft und realisiert, für die aber ein Problem gar nicht besteht oder die für die Lösung der tatsächlichen Probleme nicht geeignet sind. Trotz einer möglicherweise brillanten Lösung wäre dann doch eine schlechte Bewertung der Lösung und damit der Arbeit zu erwarten.
Aus diesem Grund sollte das Kapitel 1.1 ausführlich sein. Ein Umfang von weniger als drei Seiten wird in der Regel nicht ausreichen.

\subsection{Gegenstand}

\begin{itemize}

\item Welche Situation liegt vor und was soll getan werden?
\item Worum geht es eigentlich?
\item In welcher Welt/Domäne oder welchem Arbeitsbereich/-gebiet bewegen wir uns im Rahmen der Arbeit

\end{itemize}

\subsection{Problematik}
\begin{itemize}
\item Warum ist die geschilderte vorliegende Situation problematisch?
\item Worin bestehen die Probleme?
\item Für wen ist sie problematisch?
\end{itemize}

\subsection{Motivation}

\begin{itemize}
\item Warum lohnt es sich, die genannten Probleme zu lösen?
\item Wer wird welchen Nutzen von dieser Abschlussarbeit haben?
\item Warum ist die Arbeit wichtig?
\item Wer wartet sehnlichst auf die Fertigstellung der Arbeit

\end{itemize}

\section{Problemstellung}

Vermeiden Sie Formulierungen wie \glqq Es ist nicht bekannt, ob...\grqq oder \glqq Es existiert kein...\grqq.

Solche Formulierungen kehren in der Regel einfach das bereits angedachte Lösungsmodell um und postulieren das Fehlen der angedachten Lösung einfach als Problem. Das ist ähnlich, wie wenn es  in der Werbung hieße ?Wenn Sie das Problem haben, dass Ihnen Aspirin fehlt, dann kaufen Sie doch Aspirin?. Sinnvoller ist diese Aussage:  ?Wenn Sie das Problem haben, dass Ihnen der Kopf weh tut, dann kaufen Sie doch Aspirin?. Es ist also bei der Problembeschreibung erforderlich, sich in die Lage dessen zu versetzen, den man mit der angedachten Lösung beglücken möchte. Sein Problem ist zu ermitteln und so zu formulieren, er/sie das Problem wiedererkennt und dadurch geneigt ist, sich für die Lösung des Problems zu interessieren.

\begin{itemize}
\item Welche der in der Problematik geschilderten Probleme sollen im Rahmen dieser Abschlussarbeit gelöst werden?
\end{itemize}
Bitte jedes Einzelproblem nummerieren und mit 1-2 Sätzen beschreiben:

\begin{itemize}


\item Problem P1: Problemname 1 mit kurzer Problembeschreibung
\item Problem P2: Problemname 2 mit kurzer Problembeschreibung
\end{itemize}

\section{Zielsetzung}

\begin{itemize}
\item Welche Ergebnisse werden mit dieser Abschlussarbeit angestrebt und welche der o.g. Probleme sollen damit jeweils gelöst werden?
\end{itemize}
Bitte jedes Ziel kurz oder ggf. mit Stichworten beschreiben:
\begin{itemize}
\item Ziel(e)/angestrebte(s) Ergebnis(se) zur Lösung von Problem P1:
	\begin{itemize}
	\item Ziel Z1.1: ....
	\item Ziel Z1.2: ....
	\end{itemize}
\end{itemize}

\section{Aufgabenstellung}

\begin{itemize}
\item Wie sollen die o.g. Ziele erreicht werden?
\item Was soll zur Erreichung der Ziele bzw. zur Schaffung der Ergebnisse getan werden?
\item Welche Fragen müssen zur Erreichung der Ziele bzw. zur Schaffung der Ergebnisse beantwortet  werden?
\end{itemize}


Bitte geben Sie zu jedem der o.g. Ziele mindestens zwei Aufgaben bzw. Fragen an, die bearbeitet bzw. beantwortet werden sollen. Bitte jede Aufgabe bzw. Frage kurz oder ggf. mit Stichworten beschreiben:

\begin{itemize}
\item Aufgaben zu Ziel Z1.1:
	\begin{itemize}
	\item Aufgabe A1.1.1: ....
	\item Aufgabe A1.1.2: ....
	\end{itemize}
\end{itemize}

\section{Aufbau der Arbeit}
