%************************************************
\chapter{Einleitung}\label{ch:introduction}
%************************************************

\section{Gegenstand}

Medizininformatik ist ein Masterstudiengang an der Fakultät der Mathematik und Informatik und der Medizinischen Fakultät der Universität Leipzig.
Während der Corona-Pandemie\footnote{Sommersemester 2020 bis einschließlich zum Wintersemester 2021/2022.} wurden die Vorlesungen dieses Studiengangs (damals Schwerpunktes) von Prof. Dr. Alfred Winter online über Big Blue Button gehalten.
Von diesen Vorlesungen liegen Aufnahmen der Präsentationen und Webcams, Audioaufnahmen und Chats vor.


%- Masterstudiengang Medizininformatik (Damals Informatik Schwerpunkt)
%- Fakultät der Mathematik und Informatik und Medizinische Fakultät der Universität Leipzig
%- von prof. dr. Alfred Winter
%- Wegen Corona über BBB gehalten (Sommersemester 2020 (genauer ab 4.6) bis einschließlich Wintersemester 21/22) 
%- Vorlesungsaufnahmen (Presentationen, Audio, Webcams, Chat) 

%- Sommersemester 2021 Modul: "Architektur von Informationssystemen im Gesundheitswesen" am \ac{imise} 
%- Modul gehört zum Querschnittsbereich 1 Epidemiologie, medizinische Biometrie und medizinische Informatik 

%- Aufnahmen liegen zwar vor, Inhalt ist aber schlecht zugänglich -> Studenten brauchen eine Möglichkeit schnell und einfach auf dieses Wissen zuzugreifen
%- Deswegen sollen Aufnahmen transkibiert werden, und über ein großes %generatives Sprachmodell den Studenten zur verfügung stehen 

\section{Problemstellung}

%- Aktuelle Möglichkeiten der Studierenden: Aufnahme -> nicht Verfügbar, Zeitaufwendig, Datenschutzprobleme; Buch -> Weniger Inhalt
Deren Inhalt ist jedoch schlecht zugänglich.

Aktuell haben Studierende der medizinischen Informatik wenige Möglichkeiten zur Beantwortung von spezifischen Fragen.
Bücher und Mitschriften verfügen nicht über alle Inhalte und müssen lange nach Antworten durchsucht werden.
Oft müssen dann andere Studierende oder Professoren zur Hilfe herangezogen werden, was auch nicht immer möglich ist.

%\begin{itemize}
%\item 
Problem: Antworten auf fachspezifische Fragen zum Studiengang Medizininformatik finden, ist umständlich und zeitaufwendig.
%\item Problem P2: Bei der Veröffentlichung von Material muss das Institut Datenschutzrichtlinien~\citep{parlament2016verordnung} einhalten.
%Europäische Datenschutz Grundverordnung, Bundesdatenschutzgesetz, Landesdatenschutzgesetz 
%\item Problem P3: Aufnahmen stehen Studierenden nicht zur Verfügung
%\end{itemize}
\if false
\paragraph{Vermeiden Sie Formulierungen wie \enquote{Es ist nicht bekannt, ob$\ldots$} oder \enquote{Es existiert kein$\ldots$}.}
Solche Formulierungen kehren in der Regel einfach das bereits angedachte Lösungsmodell um und postulieren das Fehlen der angedachten Lösung einfach als Problem.
Das ist ähnlich, als wenn es in der Werbung hieße \blockquote{Wenn Sie das Problem haben, dass Ihnen Aspirin fehlt, dann kaufen Sie doch Aspirin}.
Das Problem wäre dann schon gelöst, wenn Sie Aspirin gekauft haben.
Sinnvoller ist diese Aussage:
\blockquote{Wenn Sie das Problem haben, dass Ihnen der Kopf weh tut, dann kaufen Sie doch Aspirin.}
Es ist also bei der Problembeschreibung erforderlich, sich in die Lage dessen zu versetzen, den man mit der angedachten Lösung \enquote{beglücken} möchte.
Sein Problem ist zu ermitteln und so zu formulieren, dass er/sie das Problem wiedererkennt und dadurch geneigt ist, sich für die Lösung des Problems zu interessieren.
Bei der zweiten Problembeschreibung wäre das Problem im Übrigen erst gelöst, wenn die Kopfschmerzen weg sind.


Hierzu noch ein Beispiel aus einer wissenschaftlichen Arbeit:
\blockquote{Problem: Es ist keine ganzheitliche Vorgehensweise bekannt, die die Entwicklung und Verbesserung von Software zur Steigerung der Motivation durch Unterhaltung in der Therapie auf strukturierte Art und Weise unterstützt.}
Als Ziel wird dort formuliert:
\blockquote{Erstellung einer ganzheitlichen Vorgehensweise zur Entwicklung von Software zur Steigerung der Motivation durch Unterhaltung in der Therapie.}
Es ist völlig unklar, wer das Problem hat, dass ihm oder ihr keine Vorgehensweise bekannt ist.
Außerdem wäre das Problem schon gelöst, wenn irgendwie eine ganzheitliche Vorgehensweise bekannt würde.
Tatsächlich schließt diese Arbeit auch einfach mit der Präsentation einer einheitlichen Vorgehensweise und das geschilderte Problem ist gelöst ohne noch weiter zu untersuchen, ob die Vorgehensweise irgendeinen Nutzen erbringt.
Vermutlich liegt das Problem aber etwas tiefer.
Es scheint doch wohl so zu sein, das bislang völlig unbrauchbare Software entwickelt wurde und man die Hoffnung hat, durch eine ganzheitliche Vorgehensweise bessere Ergebnisse erzielen zu können.
Das Problem sollte also eher so formuliert werden:
\blockquote{Problem: Wie aktuelle Publikationen zeigen [12-17], erfüllen die zur Zeit am Markt angebotenen  Softwareprodukte zur Steigerung der Motivation durch Unterhaltung in der Therapie nicht ihren Zweck.
So wird zum Beispiel der UntGeneratives Sprachmodellerhaltungswert von 73\% der Anwender als gering eingestuft und eine Motivationssteigerung für die Therapie konnte in keiner Studie nachgewiesen werden.}
Dazu würde folgendes Ziel passen:
\blockquote{Ziel ist eine ganzheitlichen Vorgehensweise zur Entwicklung von Software zur Steigerung der Motivation durch Unterhaltung in der Therapie, durch deren Anwendung der Unterhaltungswert für die Anwender und eine Motivationssteigerung für die Therapie erreicht werden kann.}
Das hier geschilderte Problem ist jetzt ein Problem von Patient:innen.
Bei diesem Problem und diesem Ziel muss die Arbeit daher damit schließen, dass zumindest an einem Beispiel gezeigt wird, dass die neue Vorgehensweise tatsächlich zu gesteigertem Unterhaltungswert und zur Motivationssteigerung bei Patient:innen führt.
Man könnte aber auch ‚die Latte etwas niedriger hängen‘ und sich zunächst mit folgendem Problem befassen:
\blockquote{Problem: Bei der Entwicklung von Softwareprodukten zur Steigerung der Motivation durch Unterhaltung in der Therapie übersehen viele Entwickler:innen geeignete Möglichkeiten und Spielelemente, die zur Steigerung der Motivation von Patienten eingesetzt werden können oder setzen Spielelemente für die falschen Zwecke oder an der falschen Stelle ein.}
Dazu würde dann folgendes Ziel passen:
\blockquote{Ziel ist eine ganzheitlichen Vorgehensweise zur Entwicklung von Software zur Steigerung der Motivation durch Unterhaltung in der Therapie, die die Softwarentwickler:in ausgehend von den zu erreichenden Zielen systematisch bei der Auswahl geeigneter Spielelemente zur Steigerung von Motivation unterstützt.}
In diesem Fall wird ein Problem vom Softwareentwickler:innen beschrieben und das angestrebte Ziel wird auch diese Entwickler:innen unterstützen.
Hier müsste die Arbeit damit schließen, dass man zeigt, dass es bei dem Entwickeln der Software leichter wird, Elemente so einzusetzen, dass sie dem intendierten Zweck dienen.
\fi

\section{Motivation}

Studenten brauchen eine Möglichkeit, auf dieses Wissen zuzugreifen. 

\begin{itemize}
\item Warum lohnt es sich, die genannten Probleme zu lösen?
\item Wer wird welchen Nutzen von dieser Abschlussarbeit haben?
\item Warum ist die Arbeit wichtig?
\item Wer wartet sehnlichst auf die Fertigstellung der Arbeit
\end{itemize}

Wenn ein gutes Transkript vorliegt können Studierende, mittels \acp(llm), zu jeder Zeit schnell Antworten auf konkrete Fragen finden, was sonst sehr aufwändig wäre. Dadurch könnte Lernfortschritt beschleunigt werden.


Ohne ausreichende Gegenstands-, Problem- und Motivationsbeschreibung kann eine Leser:in nicht verstehen, warum z.\,B. eine entwickelte Software sinnvoll ist bzw. zur Lösung welches Problems sie verwendet werden soll.
Gerade für die Medizinische Informatik als eine problemorientierte Disziplin ergibt sich der Wert einer Lösung, z.\,B. einer Software, aber vor allem daraus, ob bzw. wie weit sie ein Problem löst.

Für die Autor:in bedeutet daher eine unzureichende Gegenstands-, Problem- und Motivationsbeschreibung die Gefahr, dass sie oder er sich die zu lösende Problematik nicht ausreichend klar gemacht hat.
Bei der Erstellung der Arbeit besteht dann die Gefahr, dass man möglicherweise methodisch auGeneratives Sprachmodellfregende Lösungen entwirft und realisiert, für die aber ein Problem gar nicht besteht oder die für die Lösung der tatsächlichen Probleme nicht geeignet sind.
Trotz einer möglicherweise brillanten Lösung wäre dann doch eine schlechte Bewertung der Lösung und damit der Arbeit zu erwarten.
Außerdem sollte sich -- auch bei einer Abschlussarbeit -- die Arbeit auch lohnen, d.h. es sollte genügend Motivation geben, viel Zeit und Energie zu investieren.
Aus diesem Grund sollten die Kapitel 1.1 bis 1.3 ausführlich sein.
Ein Umfang von weniger als drei Seiten wird in der Regel nicht ausreichen.
Mit einem Augenzwinkern hier noch Motivationen, die wir nicht gerne in einer Abschlussarbeit sehen:\\
~~\\

\begin{tabulary}{0.965\textwidth}{LL}
Warum lohnt es sich, die genannten Probleme zu lösen?						&\enquote{Weil ich mein Studium endlich hinter mir haben will}, \enquote{Damit ich eine gute Note habe.}, \enquote{Weil Prof. Winter/mein Betreuer es so will.}\\
Wer wird welchen Nutzen von dieser Abschlussarbeit haben?					&\enquote{Ich, weil ich dann endlich mit studieren fertig bin / in den Master darf.}\\
Warum ist die Arbeit wichtig?												&\enquote{Weil mein Betreuer es möchte.}, \enquote{Weil es X noch nicht gibt.}, \enquote{Weil es X nur mit Technik Y gibt, aber nicht mit Z}, \enquote{Weil die Vorarbeiten so schlecht sind}\\
Wer wartet sehnlichst auf die Fertigstellung der Arbeit?					&\enquote{Ich / Mein Betreuer / Prof. Winter / meine Oma} (außer es ist eine Eigenentwicklung für speziell diese Personen)\\
\end{tabulary}

\section{Zielsetzung}\label{sec:zielsetzung}

\begin{itemize}
\item Ziel(e)/angestrebte(s) Ergebnis(se) zur Lösung des Problems:
	\begin{itemize}
	\item Ziel Z1: Fehlerfreies Transkript nur der fachlichen Inhalte unter Einhaltung der Datenschutzbestimmungen~\citep{parlament2016verordnung}
	\item Ziel Z2: Beantwortung von Fragen mittels \acp{llm}
	\end{itemize}

\end{itemize}

\section{Aufgabenstellung}

Deswegen sollen die Aufnahmen transkribiert werden, und über ein \ac{llm} zur Verfügung stehen.

\begin{itemize}
\item Wie sollen die o.\,g. Ziele erreicht werden?
\item Was soll zur Erreichung der Ziele bzw. zur Schaffung der Ergebnisse getan werden?
\item Welche Fragen müssen zur Erreichung der Ziele bzw. zur Schaffung der Ergebnisse beantwortet  werden?
\end{itemize}

\begin{itemize}
\item Aufgaben zu Ziel Z1:
	\begin{itemize}
	\item Aufgabe A1.1: Sichtung der Aufnahmen: Welche Module? Welche Semester? Vollständig? Änderungen?
	\item Aufgabe A1.2: Relevante Module auswählen
	\item Aufgabe A1.3: Beschaffung und Anpassung der Aufnahmen
	\item Aufgabe A1.4: Transkriptionsmodell und Parameter auswählen
	\item Aufgabe A1.5: Transkript erstellen
	\item Aufgabe A1.6: Transkript säubern
	\item Aufgabe A1.7: Transkipt Anonymisieren/Pseudonymisieren
	\end{itemize}
\item Aufgaben zu Ziel Z2:
	\begin{itemize}
	\item Aufgabe A2.1: Auwahl einer Methode zur Beantwortung von Fragen mithilfe eine Sprachmodells 
	\item Aufgabe A2.2: Auswahl eines Sprachmodells
	\item Aufgabe A2.3: Exemplarische Ausführung der Methode auf das Transkript
	\end{itemize}
\end{itemize}
